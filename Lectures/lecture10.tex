\ProvidesFile{lecture10.tex}[Лекция 10]


\begin{definition}
[Подполе]
Пусть $K$ и $L$ -- два поля причем $K \subseteq L$.
Тогда $K$ называется подполем в $L$, если операции сложения и умножения из $L$ ограниченные на $K$ дают сложение и умножение на $K$ соответственно.
Более подробно, пусть $+_L$ и $\cdot_L$ -- сложение и умножение на $L$, а $+_K$ и $\cdot_K$ -- сложение и умножение на $K$.
Тогда для любых элементов $x,y\in K$ верно: $x +_L y = x+_K y$ и $x\cdot_L y = x \cdot_K y$.
\end{definition}

По простому подполе -- это подмножество чисел в нашем поле, которое само является полем относительно операций из большего поля.
То есть нет никакой разницы какие операции использовать в подполе $K$: операции из $K$ или операции из $L$, так как между ними нет разницы.

\subsection{Абстрактное определение комплексных чисел}

\begin{definition}
Пусть поле $\mathbb C$ обладает следующими свойствами:
\begin{enumerate}
\item $\mathbb R\subseteq\mathbb C$ -- подполе, т.е. поле $\mathbb C$ содержит вещественные числа и операции сложения и умножения ограничиваются на $\mathbb R$ в обычные операции сложения и умножения.%
\footnote{В терминологии ниже вложение $\mathbb R\to \mathbb C$ является гомоморфизмом полей.}

\item Для любого не константного многочлена $f\in\mathbb C[x]$ существует корень $\alpha\in \mathbb C$, т.е. $f(\alpha) = 0$.

\item Поле $\mathbb C$ является минимальным полем, удовлетворяющем предыдущим свойствам, т.е. для любого поля $F$ такого что $\mathbb R\subseteq F\subseteq \mathbb C$ если $F$ обладает двумя предыдущими свойствами, то $F = \mathbb C$.
\end{enumerate}
Тогда оно называется полем комплексных чисел.
\end{definition}

Стоит сделать важное замечание.
Из определения вообще говоря не следует, что подобное поле существует, но даже если оно и существует, то не понятно, вообще говоря, единственно ли оно.
Как можно ожидать, такое поле обязательно существует и оказывается, что оно единственно в некотором естественном смысле.
Потому идейно к определению выше надо относиться так: это набросок тех свойств, которые мы хотели бы получить от нашего поля, а дальше вся работа заключается в том, чтобы показать, во-первых, что таких свойств добиться можно и построить необходимое поле, а во-вторых, что как бы мы ни построили поле комплексных чисел, всегда получится одно и то же.

Как вы понимаете построить поле удовлетворяющее свойству (2) -- не простая задача.
Потому обычно поступают по-другому.
Мы построим поле с более слабым свойством, что существует решение только у уравнения $x^2 + 1 = 0$.
А уже потом покажем, что подобное поле удовлетворяет более сильному условию (2) из определения.
Единственность мы с вами доказывать не будем в силу того, что эта тема будет затронута в курсе алгебры в общем виде.

\subsection{Две модели комплексных чисел}
\label{subsection::ComplexModels}

В этом разделе я построю две модели комплексных чисел.
Для того чтобы различать эти модели, я для начала буду обозначать их $\mathbb C_1$ и $\mathbb C_2$.
Но как только мы поймем, что это одно и то же, мы будем опускать индекс и обозначать построенное поле через $\mathbb C$.

\paragraph{Символьная модель}

Пусть $\mathbb C_1$ -- это множество картинок вида $a+bi$, где $a,b\in\mathbb R$ -- вещественные числа, а $i$ и $+$ -- картинки.
Множество мы определили, теперь надо определить операции сложения и умножения.
Сумму картинок определим покомпонентно:
\[
(a+bi) + (c+di) = (a+c) + (b+d) i, \quad\text{где}\quad a,b,c,d\in\mathbb R
\]
Умножение определим исходя из соображений $i^2 = -1$.
Тогда
\[
(a+bi)(c+di) = (ac - bd) + (ad + bc)i,\quad\text{где}\quad a,b,c,d\in\mathbb R
\]
На этом этапе необходимые данные для определения поля нами построены.
Осталось дело за малым -- проверить все 10 аксиом.
Эту наинтереснейшую задачу я оставлю в качестве упражнения, но обязательно проверьте эти аксиомы.

Теперь $\mathbb C_1$ является полем.
Вещественные числа в него вкладываются так: число $r\in\mathbb R$ идет в картинку $r + 0 i\in\mathbb C_1$.
Теперь надо проверить что сложить два вещественных числа -- это все равно, что сложить их как два комплексных числа.
Аналогично, умножить два вещественных числа -- это все равно, что умножить их как два комплексных числа.
Напоследок заметим, что число комплексное число $i$ выбрано так, чтобы оно являлось решением уравнения $x^2 + 1 = 0$.

\paragraph{Матричная модель}

Пусть $\mathbb C_2$ -- это множество матриц вида $\left(\begin{smallmatrix}{a}&{-b}\\{b}&{a}\end{smallmatrix}\right)\in\Matrix{2}$.
Множество построено, теперь дело за операциями.
Придумывать их не надо, это будут обычные матричные сложение и умножение, единственное, что надо проверить, что сумма и произведение матриц из $\mathbb C_2$ остаются в $\mathbb C_2$.%
\footnote{Это, пусть и легкое, упражнение мы оставляем на совести читателя.} 

Как и с первой моделью, мы только что построили все необходимые данные для определения поля, теперь надо проверить аксиомы.
И тут нам очень пригождаются матрицы.
Почти все аксиомы будут автоматически следовать из соответствующих свойств матричных операций.
Единственное, что надо проверить: коммутативность умножения и что любой ненулевой элемент обратим.
Я сейчас опять поступлю не очень честно и попрошу жаждущего до знаний читателя все проверить самостоятельно.%
\footnote{На самом деле я всего лишь полу-честен с вами ибо чуть ниже будут проведены все соответствующие проверки.}

Вещественные числа вкладываются в $\mathbb C_2$ в виде скалярных матриц, то есть $r\in\mathbb R$ идет в $rE\in\mathbb C_2$.
Как мы знаем при этом операции матричного сложения и умножения превращаются в операции сложения и умножения вещественных чисел.
Осталось найти решение уравнения $x^2+1 = 0$.
Для этого заметим, что матрица $\left(\begin{smallmatrix}{0}&{-1}\\{1}&{0}\end{smallmatrix}\right)$ удовлетворяет этому уравнению.

\paragraph{Сравнение полей}

Для того чтобы сравнить различные поля и сказать, что они одинаковые или различные нам потребуется понятие изоморфизма полей.

\begin{definition}
Пусть $F_1$ и $F_2$ поля.
Отображение $\varphi\colon F_1\to F_2$ называется гомоморфизмом полей, если 
\begin{enumerate}
\item $\varphi(x+y) = \varphi(x) + \varphi(y)$ для любых $x,y\in F_1$.

\item $\varphi(xy) = \varphi(x)\varphi(y)$ для любых $x,y\in F_1$.

\item $\varphi(1) = 1$.
\end{enumerate}
Если $\varphi$ является биекцией, то оно называется изоморфизмом.
\end{definition}

Стоит отметить, что гомоморфизм полей всегда инъективен.
Попробуйте доказать это.
Кроме того, если отображение $\varphi$ инъективно, то достаточно лишь проверить первые два свойства гомоморфизма, т.е. единица автоматически перейдет в единицу.%
\footnote{Догадливый читатель уже сообразил, что в этом месте будет фраза: ``Проверьте это''.}

Мы будем говорить что два поля изоморфны, если между ними существует изоморфизм.
Про изоморфные поля надо думать, как про одинаковые поля.
Действительно, что значит, что между множествами есть биекция.
Это значит, что это на самом деле одно и то же множество, а биекция лишь переопределяет имена, которыми называются наши элементы.
Изоморфизм кроме всего прочего сохраняет операции, это значит, что отождествив элементы наших полей, мы не различаем проделанных операций.
Так как поле для нас -- это множество с операциями, то значит мы не увидим никакой разницы, между полями, если в них одинаковые операции.

\begin{problems}
Пусть $\varphi\colon K\to L$ -- гомоморфизм полей.
Покажите, что выполнены следующие вещи:
\begin{enumerate}
\item $\varphi(0) = 0$.

\item $\varphi(-x) = -\varphi(x)$ для любого $x\in K$.

\item Если в определении гомоморфизма оставить только свойства $1$ и $2$, то $\varphi(1)$ либо $0$, либо $1$.
В частности, если $\varphi$ инъективно, то $\varphi(1) = 1$.

\item $\varphi(x^{-1}) = \varphi(x)^{-1}$ для любого ненулевого $x\in K$.
\end{enumerate}
\end{problems}

\paragraph{Сравнение моделей комплексных чисел}

Прежде чем объяснить, что $\mathbb C_1$ и $\mathbb C_2$ -- это одно и то же.
Нам понадобится еще одно определение.
Дело в том, что в наших полях лежат дополнительно вещественные числа и мы, когда будем сравнивать эти два поля, хотим чтобы это сравнение было согласовано в каком-то смысле с вещественными числами.

\begin{definition}
Пусть $F_1$ и $F_2$ -- два поля, содержащие поле вещественных чисел $\mathbb R$, то есть $\mathbb R\subseteq F_1$ и $\mathbb R\subseteq F_2$ и операции с $F_i$ ограничиваются на соответствующие операции на вещественных числах.
Будем говорить, что $\varphi\colon F_1\to F_2$ является изоморфизмом над $\mathbb R$, если 
\begin{enumerate}
\item $\varphi$ является изоморфизмом.

\item $\varphi(r) = r$ для любого вещественного числа $r\in \mathbb R$.
\end{enumerate}
\end{definition}

Давайте построим изоморфизм над $\mathbb R$ между $\mathbb C_1$ и $\mathbb C_2$.
А именно: $\varphi\colon \mathbb C_1\to \mathbb C_2$ будет действовать по правилу $a+bi \mapsto \left(\begin{smallmatrix}{a}&{-b}\\{b}&{a}\end{smallmatrix}\right)$.
По построению очевидно, что данное отображение является биекцией.
Кроме того, очевидно, что оно переводит сумму в сумму.
Методом пристального взгляда проверяем, что $\varphi$ сохраняет умножение.
Вещественное число $r = r + 0i$ переходит в матрицу $\left(\begin{smallmatrix}{r}&{0}\\{0}&{r}\end{smallmatrix}\right) = rE$.
С учетом нашего отождествления вещественных чисел с подмножествами в $\mathbb C_1$ и $\mathbb C_2$ последнее означает, что $\varphi(r) = r$ для любого $r\in\mathbb R$.
То есть нет никакой разницы между этими двумя моделями.
Причем на столько нет разницы, что при нашем отождествлении все новые числа в одной модели имеют ровно те же отношения со старыми числами, что и в другой (это по сути философия изоморфизма над $\mathbb R$).
С этого момента мы будем обозначать любую из этих двух моделей через $\mathbb C$.

\subsection{Простейшие свойства и операции}

\paragraph{Комплексное сопряжение}

Определим следующую операцию $\bar{\phantom{z}}\colon \mathbb C\to \mathbb C$ по правилу $z = a + bi \mapsto \bar z = a - bi$.
На языке матричной модели эта операция соответствует транспонированию.
Мы знаем, что транспонирование переводит сумму в сумму, а на произведении действует так $(AB)^t = B^t A^t$, но так как $\mathbb C_2$ коммутативно, то для матриц из $\mathbb C_2$ мы имеем $(AB)^t = A^t B^t$.
Кроме того сопряжение биективно, как видно из построения и переводит вещественные числа в вещественные.
Значит сопряжение является изоморфизмом $\mathbb C$ на $\mathbb C$ над $\mathbb R$.

Сделаем еще одно полезное замечание, на матричном языке сопряжение так же совпадает с вычислением присоединенной матрицы.
Действительно,
\[
\begin{pmatrix}
{a}&{-b}\\
{b}&{a}
\end{pmatrix}^t
= 
\begin{pmatrix}
{a}&{b}\\
{-b}&{a}
\end{pmatrix}
=
\widehat{
\begin{pmatrix}
{a}&{-b}\\
{b}&{a}
\end{pmatrix}}
\]

У последнего замечания есть интересное философское следствие.
Заметим, что сопряжение переводит $i$ в $-i$.
А так как оно является изоморфизмом, то это означает, что между $i$ и $-i$ нет никакой разницы.
То есть если мы внезапно обозначим $-i$ за $j$, то $i$ превратится в $-j$ и все комплексные числа будут иметь вид $a + bj$ и в этой новой форме никто не догадается, что $j$ это была $-i$, а не опечатка наборщика перепутавшего буквы $i$ и $j$.
То есть в поле комплексных чисел есть небольшая свобода выбора.
Мы случайно выбрали один из корней уравнения $x^2 + 1 = 0$ за $i$ и на самом деле нет никакой разницы какой из его корней мы так обозначим.

\paragraph{Вещественная и мнимая части}

Когда комплексное число записано в виде $z = a + bi$, где $a,b\in\mathbb R$, мы говорим, что это его алгебраическая форма.
В таком случае число $a$ называется его вещественной частью и обозначается $\Re z$, а $b$ называется мнимой частью $z$ и обозначается $\Im z$.
Числа с нулевой мнимой частью -- это вещественные числа, а числа с нулевой вещественной частью называются чисто мнимыми.

Заметим, что для любого числа $z\in\mathbb C$ верно
\begin{enumerate}
\item $z\in \mathbb R$ тогда и только тогда, когда $\bar z= z$.
\item $z\in i\mathbb R$ тогда и только тогда, когда $\bar z = - z$.
\end{enumerate}

\subsection{Геометрическая модель}

Комплексные числа $\mathbb C$ можно отождествить с вещественной плоскостью $\mathbb R^2$, а именно $a+bi$ соответствует вектору на плоскости $\left(\begin{smallmatrix}a\\b\end{smallmatrix}\right)$.
Таким образом про каждое комплексное число можно думать геометрически как про вектора.
При этом сложение комплексных чисел соответствует сложению векторов на плоскости.

У каждого вектора есть длина $|z| = \sqrt{a^2 + b^2}$ -- эта величина называется модулем комплексного числа.
В матричной модели у нас определитель, легко увидеть, что $\det z = |z|^2$.
Перечислим свойства модуля в следующем утверждении.

\begin{claim*}
Модуль комплексного числа обладает следующими свойствами:
\begin{enumerate}
\item $|{-}|\colon \mathbb C\to \mathbb R_+$ является нормой, то есть 
\begin{itemize}
\item $|z| \geqslant 0$ для любого $z\in\mathbb C$, причем равенство нулю достигается тогда и только тогда, когда $z = 0$.

\item $|\lambda z| = |\lambda | |z|$ для любого $\lambda \in \mathbb R$ и $z\in\mathbb C$.

\item $|z + w|\leqslant |z| + |w|$ для любых $z,w\in\mathbb C$.
\end{itemize}

\item $z\bar z = |z|^2$ для любого $z\in\mathbb C$.

\item $|zw| = |z| |w|$ для любых $z,w\in\mathbb C$.

\item $z^{-1} = \frac{\bar z}{|z|^2}$.
\end{enumerate}
\end{claim*}
\begin{proof}
Проверку (1) я оставлю на совести читателя.
(2) -- это явная формула.
(3) доказывается с использованием (2).
А вот (4) -- это явная формула для обратной матрицы, потому что в матричной модели $\bar z$ -- это сопряженная матрица, а $|z|^2$ -- это $\det (z)$.
\end{proof}

\paragraph{Тригонометрическая форма}

Пусть $z\in\mathbb C$ и пусть $z\neq 0$.
Тогда мы можем сделать следующее
\[
z = a + bi = \sqrt{a^2 + b^2}\left(\frac{a}{\sqrt{a^2 + b^2}} + \frac{b}{\sqrt{a^2 + b^2}}i\right)
\]
У числа в скобках вещественная и мнимая часть после возведения в квадрат в сумме дают единицу, а значит они являются косинусом и синусом некоторого числа $\varphi$, а значит, $z$ можно переписать в следующей форме
\[
z = |z| (\cos \varphi + i\sin\varphi)
\]
Такая запись комплексного числа называется тригонометрической.
Число $\varphi$ определено с точностью до $2\phi n$, $n\in\mathbb Z$ и называется аргументом комплексного числа.
Геометрически $\varphi$ -- это угол между осью $OX$ и вектором проходящим из нуля в $z$.
Угол отсчитывается против часовой стрелки.
Существует следующее удобное соглашение
\[
e^{i\varphi} = \cos \varphi + i \sin \varphi
\]
Оказывается, что при таком определении экспонента обладает всеми знакомыми нам свойствами.%
\footnote{Если заглянуть чуть глубже в большую науку, то окажется, что вас ждет некоторый набор чудес.
Окажется, что в комплексном мире очень мало гладких функций и они очень жесткие.
Это значит, что для любой вещественной гладкой функции $f\colon \mathbb R\to \mathbb R$ существует не более одной комплексной гладкой функции $\tilde f\colon \mathbb C\to \mathbb C$, продолжающей $f$, в том смысле, что $\tilde f(r) = f(r)$ для любой $r\in \mathbb R$.
Потому как бы мы не продолжили нашу вещественную экспоненту в комплексный мир, все эти способы дают одно и то же.}
В этом случае тригонометрическую форму можно записать так
\[
z = |z| e^{i\varphi} = e^{\ln |z| + i\varphi}
\]

Алгебраическая форма записи комплексного числа хорошо согласована со сложением, а тригонометрическая -- с умножением, о чем говорит следующее.

\begin{claim*}
Для комплексных числе в тригонометрической форме верны следующие формулы
\begin{enumerate}
\item Пусть $z_1 = r_1(\cos \varphi + i \sin \varphi)$ и $z_2=r_2(\cos \psi + i \sin \psi)$ -- два ненулевых комплексных числа, тогда
\[
z_1 z_2 = r_1 r_2 (\cos(\varphi + \psi) + i\sin(\varphi + \psi))
\]

\item Пусть $z_1 = r_1(\cos \varphi + i \sin \varphi)$ и $z_2=r_2(\cos \psi + i \sin \psi)$ -- два ненулевых комплексных числа, тогда
\[
\frac{z_1}{z_2} = \frac{r_1}{r_2}(\cos(\varphi - \psi) + i\sin(\varphi - \psi))
\]

\item {\bf Формулы Муавра}
Пусть $z = r(\cos \varphi + i \sin \varphi)$ ненулевое комплексное число, тогда
\[
z^n = r^n (\cos(n \varphi) + i \sin(n \varphi))
\]
\end{enumerate}
\end{claim*}
\begin{proof}
1) По определению
\begin{gather*}
r_1(\cos \varphi + i \sin \varphi) r_2 (\cos \psi + i \sin \psi) = r_1 r_2 (\cos \varphi \cos \psi - \sin \varphi \sin \psi + i (\sin\varphi \cos \psi + \sin \psi \cos \varphi)) =\\ r_1r_2(\cos(\varphi + \psi) + i\sin (\varphi + \psi))
\end{gather*}

2) Можно проверить двумя способами.
Либо воспользоваться тем, что $1/z_2 = \bar z_2 /|z_2|^2$, либо домножить требуемое равенство на $z_2$ и тогда проверка сводится к первому пункту.

3) Это непосредственное следствие первого пункта.
\end{proof}


\subsection{Основная теорема алгебры}

\begin{definition}
Поле $F$ называется алгебраически замкнутым, если для любого многочлена $f\in F[x]\setminus F$ существует корень $\alpha\in F$, то есть $f(\alpha) = 0$.
\end{definition}


\begin{claim}
[Основная теорема алгебры]
Поле $\mathbb C$ построенное в разделе~\ref{subsection::ComplexModels} алгебраически замкнуто.
\end{claim}

\paragraph{План доказательства}

Доказательство основной теоремы алгебры проведем следующим образом.
Нам надо будет показать, что верны следующие два утверждения:
\begin{enumerate}
\item Пусть $p\in \mathbb C[x]$ -- произвольный многочлен, тогда отображение $|p|\colon \mathbb C\to \mathbb R$ заданное по правилу $z\mapsto |p(z)|$ достигает минимума, то есть найдется такая точка $z_0\in\mathbb C$, что $|p(z_0)|\leqslant |p(z)|$ для любого $z\in \mathbb C$.

\item Пусть $p\in \mathbb C[x]$ -- не константный многочлен и пусть $z\in \mathbb C$ такая точка, что $p(z) \neq 0$.
Тогда найдется точка $z_1\in \mathbb C$ такая, что $|p(z_1)| < |p(z)|$.
\end{enumerate}

В начале я покажу, как из этих трех утверждений вытекает основная теорема.

\begin{proof}
[Доказательство основной теоремы алгебры]
Возьмем произвольный неконстантный многочлен $p\in\mathbb C[x]$.
Мы должны показать, что он имеет хотя бы один корень в $\mathbb C$.
Предположим противное, пусть у него нет корней.
По первому пункту мы знаем, что найдется точка минимума $z_0\in\mathbb C$ для отображения $|p|\colon \mathbb C\to \mathbb R$.
То есть для любой точки $z\in\mathbb C$ будет выполнено $|p(z_0)|\leqslant |p(z)|$.
Так как по предположению у нас нет корней у многочлена $p$, то $p(z_0) \neq 0$.
А значит по второму пункту мы можем найти точку $z_1\in\mathbb C$ такую, что $|p(z_1)|< |p(z)|$, что противоречит тому, что $z_0$ была точкой минимума.
Противоречие получилось из нашего предположения, что $p$ не имело корней.
\end{proof}

Теперь для завершения доказательства нам надо лишь показать истинность двух утверждений выше.



\begin{claim}
Пусть $p\in \mathbb C[x]$ -- произвольный многочлен, тогда отображение $|p|\colon \mathbb C\to \mathbb R$ заданное по правилу $z\mapsto |p(z)|$ достигает минимума, то есть найдется такая точка $z_0\in\mathbb C$, что $|p(z_0)|\leqslant |p(z)|$ для любого $z\in \mathbb C$.
\end{claim}
\begin{proof}
Идея доказательства этого утверждения следующая.
Пусть $c = |p(0)|$.
Если это ноль, то мы нашли наш минимум.
Пусть $c\neq 0$, тогда давайте найдем диск $D_r(0)$ с центром в нуле и радиуса $r$ такой, что $|p(z)| > c$ для всех $z\notin D_r(0)$.
Тогда, если мы найдем минимум для $|p(z)|$ на диске $D_r(0)$ он автоматически будет минимумом в $\mathbb C$.
Действительно, внутри диска в этой точке мы будем принимать наименьшее значение, в частности значение не будет больше $c$.
Но вне диска мы не можем принять значение меньше, так как там мы строго больше $c$.
Теперь у нас две задачи: 1) найти нужный диск $D_r(0)$ и 2) найти минимум внутри диска $D_r(0)$.

\paragraph{Минимум на диске}

На диске минимум искать очень просто.
Правильный путь -- это сослаться на следующий факт.
\begin{claim*}
[БД]
Пусть $E\subseteq \mathbb R^n$ -- компактное подмножество (т.е. замкнутое и ограниченное)%
\footnote{На самом деле правильное определение компакта совсем другое.
Если вы хотите открыть глаза на мир, то стоит узнать про такую науку как Топология.
Именно она занимается всякими сортами непрерывностей и другими (иногда ужасными) чудесами.}
и $\varphi\colon E \to \mathbb R$ -- непрерывная функция.
Тогда существует минимум и максимум у $\varphi$ на $E$.
\end{claim*}

Но я покажу, как этот факт вывести из более знакомого результата.
\begin{claim*}
[БД]
У любой последовательности на отрезке найдется сходящаяся подпоследовательность.
То есть для любой $a_n\in [a, b]$ найдется подпоследовательность $a_{n_k}$ такая, что существует
\[
\lim_{k\to \infty}a_{n_k}\in [a, b]
\]
\end{claim*}

Рассмотрим функцию $f = |p|\colon D_r(0)\to \mathbb R$ по правилу $f(z) = |p(z)|$.
Так как эта функция ограничена снизу нулем, то существует нижняя грань
\[
a = \inf_{z\in D_r(0)} f(z)
\]
По определению нижней грани, мы можем выбрать последовательность $z_n\in D_r(0)$ такую, что $f(z_n) \to a$.
Такая последовательность обязательно имеет вид $z_n = a_n + i b_n$, где $a_n, b_n\in [-r, r]$ -- последовательности вещественных чисел на отрезке.
Из них мы по очереди можем выбрать сходящиеся подпоследовательности $a_{n_k}$ и $b_{n_k}$, так что последовательность $z_{n_k}$ сходится в $D_r(0)$ к какой-то точке $z_0$.
А значит
\[
a = \lim_{k\to \infty} f(z_{n_k}) = \lim_{k\to \infty} |p(z_{n_k})| = \left|\lim_{k\to \infty} p(z_{n_k})\right| = \left|p\left(\lim_{k\to \infty} z_{n_k}\right)\right| = |p(z_0)|
\]
Напомню, что непрерывность функции означает, что мы можем пронести внутрь нее предел.
Третье равенство следует из непрерывности модуля, а четвертое из непрерывности многочлена.
Таким образом в точке $z_0$ достигается нижняя грань на диске $D_r(0)$, а значит это точка минимума.

\paragraph{Существование нужного диска}

Таким образом нам остается показать, что действительно найдется диск $D_r(0)$, что вне него функция $|p(z)|$ принимает значения больше любого наперед заданного числа.
Пусть $p(z) = a_0 + a_1 z + \ldots + a_n z^n$ и $a_n\neq 0$.
Тогда
\[
p(z) = a_n z^n \left(1 + \frac{a_{n-1}}{a_n z} + \ldots + \frac{a_0}{a_n z^n}\right) = a_n z^n (1 + \omega(z))
\]
Фиксируем произвольное положительное число $r > 1$ и рассмотрим $|z| > r$.
Тогда 
\[
|\omega(z)| \leqslant \left|\frac{a_{n-1}}{a_n z}\right| + \ldots + \left|\frac{a_0}{a_n z^n}\right|\leqslant \left|\frac{a_{n-1}}{a_n }\right|\frac{1}{r} + \ldots + \left|\frac{a_0}{a_n }\right|\frac{1}{r^n}\leqslant  \left(\left|\frac{a_{n-1}}{a_n }\right| + \ldots + \left|\frac{a_0}{a_n }\right|\right)\frac{1}{r} 
\]
Последнее выражение  обозначим за $\delta(r)$, оно идет к нулю при $r\to \infty$.
Давайте теперь оценим вне этого диска значение $|p(z)|$:
\[
|p(z)| = |a_n z^n(1+\omega(z))| =|a_n| |z|^n |1 + \omega(z)|\geqslant |a_n| |z|^n (1 - |\omega(z)|)\geqslant |a_n| |r|^n(1 - \delta(r))\to \infty,\text{ при } r\to \infty
\]
То есть мы сможем найти $r$ при котором вне диска $D_r(0)$ будет выполняться $|p(z)| > c$.
\end{proof}
