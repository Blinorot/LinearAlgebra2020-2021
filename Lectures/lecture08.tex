\ProvidesFile{lecture08.tex}[Лекция 8]


\subsection{Миноры и алгебраические дополнения}

\paragraph{Определения}

Пусть $B\in\Matrix{n}$ -- некоторая матрица с $b_{ij}$.
Рассмотрим матрицу $D_{ij}\in\Matrix{n-1}$ полученную из $B$ вычеркиванием $i$-ой строки и $j$-го столбца.
Определитель матрицы $D_{ij}$ обазначается $M_{ij}$ и называется {\it минором} матрицы $B$ или $i\,j$-минором для определенности.
Число $A_{ij} = (-1)^{i+j}M_{ij}$ называется {\it алгебраическим дополнением} элемента $b_{ij}$ или $i\,j$-алгебраическим дополнением матрицы $B$.

Покажем как это все выглядит на картинках.
Если мы представим матрицу $B$ в виде
\[
B =
\left(
\begin{array}{c|c|c}
\cline{2-2}
{X_{ij}}&{
\begin{array}{c}
{*}\\{\vdots}
\end{array}
}&{Y_{ij}}\\
\hline
\multicolumn{1}{|c|}{
\begin{array}{cc}
{*}&{\ldots}
\end{array}
}&{b_{ij}}&\multicolumn{1}{c|}{
\begin{array}{cc}
{\ldots}&{*}
\end{array}
}\\
\hline
{Z_{ij}}&{
\begin{array}{c}
{\vdots}\\{*}
\end{array}
}&{W_{ij}}\\
\cline{2-2}
\end{array}
\right)
\]
Тогда
\[
D_{ij} =
\begin{pmatrix}
{X_{ij}}&{Y_{ij}}\\
{Z_{ij}}&{W_{ij}}\\
\end{pmatrix},\quad
M_{ij} = 
\det
\begin{pmatrix}
{X_{ij}}&{Y_{ij}}\\
{Z_{ij}}&{W_{ij}}\\
\end{pmatrix}\quad\text{и}\quad
A_{ij} =
(-1)^{i+j}
\det
\begin{pmatrix}
{X_{ij}}&{Y_{ij}}\\
{Z_{ij}}&{W_{ij}}\\
\end{pmatrix}
\]
{\it Присоединенная матрица} $\hat B$ для $B$ определяется как
\[
\hat B = 
\begin{pmatrix}
{A_{11}}&{A_{21}}&{\ldots}&{A_{n1}}\\
{A_{12}}&{A_{22}}&{\ldots}&{A_{n2}}\\
{\vdots}&{\vdots}&{\ddots}&{\vdots}\\
{A_{1n}}&{A_{2n}}&{\ldots}&{A_{nn}}\\
\end{pmatrix}
\]
То есть надо в матрице $B$ каждый элемент $b_{ij}$ заменить на его алгебраическое дополнение $A_{ij}$, а потом полученную матрицу транспонировать.
Полезно держать перед глазами формулу для элемента присоединенной матрицы $\hat B_{ij} = A_{ji}$.

\paragraph{Формула разложения по строке}

\begin{claim}
\label{claim::DetExpand}
Пусть $B\in\Matrix{n}$ -- произвольная матрица.
Тогда%
\footnote{Всюду в формулах $A_{ij}$ обозначает алгебраическое дополнение.}
\begin{enumerate}
\item Для любой строки $i$ верно разложение
\[
\det B = \sum_{j=1}^n b_{ij} A_{ij}
\]

\item Для любого столбца $j$ верно разложение
\[
\det B = \sum_{i=1}^n b_{ij} A_{ij}
\]
\end{enumerate}
\end{claim}
\begin{proof}
Мы докажем формулу для строки, для столбца она получается аналогично либо применением транспонирования к матрице.
Рассмотрим $i$-ю строку в матрице $B$
\[
B =
\left(
\begin{array}{ccc}
{X_{ij}}&{
\begin{array}{c}
{*}\\{\vdots}
\end{array}
}&{Y_{ij}}\\
\hline
\multicolumn{1}{|c}{
\begin{array}{cc}
{b_{i1}}&{\ldots}
\end{array}
}&{b_{ij}}&\multicolumn{1}{c|}{
\begin{array}{cc}
{\ldots}&{b_{in}}
\end{array}
}\\
\hline
{Z_{ij}}&{
\begin{array}{c}
{\vdots}\\{*}
\end{array}
}&{W_{ij}}\\
\end{array}
\right)
\]
Эту строку можно разложить в сумму следующих строк 
\[
(b_{i1},\ldots,b_{in}) = \sum_{j=1}^n(0,\ldots,0,b_{ij},0,\ldots,0)
\]
Теперь вычислим определитель $B$ пользуясь линейностью по $i$-ой строке
\[
\det B = 
\sum_{j=1}^n
\det
\left(
\begin{array}{c|c|c}
\cline{2-2}
{X_{ij}}&{
\begin{array}{c}
{*}\\{\vdots}
\end{array}
}&{Y_{ij}}\\
\hline
\multicolumn{1}{|c|}{
\begin{array}{cc}
{0}&{\ldots}
\end{array}
}&{b_{ij}}&\multicolumn{1}{c|}{
\begin{array}{cc}
{\ldots}&{0}
\end{array}
}\\
\hline
{Z_{ij}}&{
\begin{array}{c}
{\vdots}\\{*}
\end{array}
}&{W_{ij}}\\
\cline{2-2}
\end{array}
\right)
\]
Теперь отдельно посчитаем следующий определитель
\[
\det 
\left(
\begin{array}{c|c|c}
\cline{2-2}
{X_{ij}}&{
\begin{array}{c}
{*}\\{\vdots}
\end{array}
}&{Y_{ij}}\\
\hline
\multicolumn{1}{|c|}{
\begin{array}{cc}
{0}&{\ldots}
\end{array}
}&{b_{ij}}&\multicolumn{1}{c|}{
\begin{array}{cc}
{\ldots}&{0}
\end{array}
}\\
\hline
{Z_{ij}}&{
\begin{array}{c}
{\vdots}\\{*}
\end{array}
}&{W_{ij}}\\
\cline{2-2}
\end{array}
\right)
=
(-1)^{j-1}
\det
\left(
\begin{array}{|c|cc}
\cline{1-1}
{
\begin{array}{c}
{*}\\{\vdots}
\end{array}
}&{X_{ij}}&{Y_{ij}}\\
\hline
{b_{ij}}&{\ldots}&\multicolumn{1}{c|}{0}\\
\hline
{
\begin{array}{c}
{\vdots}\\{*}
\end{array}
}&{Z_{ij}}&{W_{ij}}\\
\cline{1-1}
\end{array}
\right)
=
(-1)^{j-1}(-1)^{i-1}
\det
\left(
\begin{array}{|c|cc}
\hline
{b_{ij}}&{\ldots}&\multicolumn{1}{c|}{0}\\
\hline
{\vdots}&{X_{ij}}&{Y_{ij}}\\
{*}&{Z_{ij}}&{W_{ij}}\\
\cline{1-1}
\end{array}
\right)
\]
В первом равенстве мы переставили $j$-ый столбец $j-1$ раз, чтобы переместить его на место первого столбца.
Во втором равенстве мы переставили $i$-ю строку $i-1$ раз, чтобы переставить ее на место первой строки.
Последняя матрица является блочно нижнетреугольной, а следовательно, равенство можно продолжить так
\[
(-1)^{i+j} b_{ij}
\det
\begin{pmatrix}
{X_{ij}}&{Y_{ij}}\\
{Z_{ij}}&{W_{ij}}\\
\end{pmatrix}
= b_{ij}(-1)^{i+j}M_{ij} = b_{ij}A_{ij}
\]
\end{proof}


\paragraph{Явные формулы для обратной матрицы}

\begin{claim}
\label{claim::InvMatExplicite}
Для любой матрицы $B\in \Matrix{n}$ верно 
\[
\hat B B = B\hat B = \det(B) E
\]
\end{claim}
\begin{proof}
Нам надо отдельно доказать два равенства $\hat B B = \det (B) E$ и $B\hat B = \det (B) E$.
Давайте докажем второе равенство, а первое показывается аналогично (или через трюк с транспонированием).

Для доказательства $B\hat B = \det (B) E$ нам надо показать две вещи:
(1) все диагональные элементы матрицы $B\hat B$ равны $\det (B)$, (2) все внедиагональные элементы равны нулю.

(1) Рассмотрим $i$ диагональный элемент в матрице $B\hat B$:
\[
(B\hat B)_{ii} = \sum_{j=1}^n b_{ij}\hat B_{ji} = \sum_{j=1}^n b_{ij}A_{ij}=\det(B)
\]
Последняя формула является разложением определителя $\det (B)$ по $i$-ой строке из утверждения~\ref{claim::DetExpand}.

(2) Рассмотрим элемент на позиции $i\,j$ для $i\neq j$:
\[
(B\hat B)_{ij} = \sum_{k=1}^n b_{ik}\hat B_{kj} = \sum_{k=1}^n b_{ik}A_{jk}
\]
Нам надо показать, что последнее выражение равно нулю.
Давайте рассмотрим матрицу $B$ и заменим в ней $j$-ю строку на $i$-ю, все остальные оставим нетронутыми.
Обозначим полученную матрицу через $B'$. Тогда
\[
B' =
\left(
\begin{array}{ccccc}
{*}&{\ldots}&{*}&{\ldots}&{*}\\
\hline
\multicolumn{1}{|c}{b_{i1}}&{\ldots}&{b_{ik}}&{\ldots}&\multicolumn{1}{c|}{b_{in}}\\
\hline
{*}&{\ldots}&{*}&{\ldots}&{*}\\
\hline
\multicolumn{1}{|c}{b_{i1}}&{\ldots}&{b_{ik}}&{\ldots}&\multicolumn{1}{c|}{b_{in}}\\
\hline
{*}&{\ldots}&{*}&{\ldots}&{*}\\
\end{array}
\right)
\]
Давайте посчитаем определитель $B'$ двумя способами.
С одной стороны $\det(B') = 0$ так как в матрице есть две одинаковые строки.
С другой стороны, давайте разложим определитель $\det(B')$ по $j$-ой строке
\[
\det (B') = \sum_{k=1}^n b_{ik}A_{jk}
\]
Что и требовалось доказать.
\end{proof}

В качестве непосредственного следствия этого утверждения получаем явные формулы обратной матрицы.%
\footnote{Заметим, что для формулы требуется условие $\det (B)\neq 0$.
Однако, матрица обратима тогда и только тогда, когда $\det(B)\neq 0$.
Один из способов это показать -- применить $\det$ к равенству $B B^{-1} = E$ и увидеть, что $\det(B) \det(B^{-1}) = 1$.
А в обратную сторону -- явные формулы.}

\begin{claim}[Явные формулы обратной матрицы]
Пусть $B\in\Matrix{n}$ -- обратимая матрица, тогда 
\[
B^{-1} = \frac{1}{\det(B)}\hat B
\]
\end{claim}

Заметим, что в случае матрицы $2$ на $2$ формулы принимают следующий вид
\[
\begin{pmatrix}
{a}&{b}\\
{c}&{d}\\
\end{pmatrix}^{-1}
=
\frac{1}{ad - bc}
\begin{pmatrix}
{d}&{-b}\\
{-c}&{a}\\
\end{pmatrix}
\]


\subsection{Формулы Крамера}

Пусть $A\in\Matrix{n}$ -- произвольная матрица и $b\in\mathbb R^n$ -- столбец.
Рассмотрим систему линейных уравнений $Ax = b$.
Давайте в матрице $A$ $i$-ый столбец заменим на $b$, а остальные столбцы оставим как есть.
Обозначим полученную матрицу через $\bar A_i$.
Определим $\Delta = \det (A)$ и $\Delta_i = \det (\bar A_i)$.

Мы знаем, что данная система имеет единственное решение для любого $b$ тогда и только тогда, когда матрица $A$ обратима.
Следующее утверждение дает явные формулы для координат решения системы в этом случае.

\begin{claim}[Формулы Крамера]
Пусть $A\in\Matrix{n}$, $x,b\in \mathbb R^n$ и выполнено равенство $Ax = b$.
Тогда $\Delta x_i = \Delta_i$ для любого $i$.%
\footnote{Здесь $x_i$ -- координаты вектора $x$.}
\end{claim}
\begin{proof}
Возьмем в матрице $A$ заменим $i$-ый столбец на столбец $b$.
Обозначим результирующую матрицу через $\bar A_i$.
Рассмотрим матрицу $A$ как строку из столбцов $A = (A_1|\ldots|A_n)$, где $A_i$ -- столбцы матрицы $A$.
Тогда равенство $Ax = b$, пользуясь блочными формулами, можно переписать так $x_1 A_1 + \ldots + x_n A_n = b$.
Давайте посчитаем определитель $A_i$, пользуясь последним равенством. 
\[
\det (\bar A_i) = \det(A_1|\ldots|b|\ldots|A_n) = \det(A_1|\ldots|\sum_{k=1}^n x_k A_k|\ldots|A_n) = \sum_{k=1}^n x_k \det (
\stackrel{i}{A_1|\ldots|A_k|\ldots|A_n})
\]
В последней формуле, если $k\neq i$, то слагаемое имеет два одинаковых столбца $A_i$.
Потому остается только одно слагаемое для $k = i$.
Получаем
\[
\det(\bar A_i) = x_i \det(A_1|\ldots|A_i|\ldots|A_n) = x_i \det(A)
\]
Что и требовалось.
\end{proof}

Заметим, что если $\Delta = \det (A) \neq 0$, то имеется единственное решение системы $Ax = b$ для любой правой части $b$ и координаты этого решения заданы по формулам $x_i =\frac{\Delta_i}{\Delta}$.
Однако, если $\Delta = \det(A) = 0$, то либо решений бесконечное число, либо их вообще нет.
В этом случае единственная информация из формул Крамера это: $\Delta_i = 0$.


\subsection{Характеристический многочлен}

Пусть $A\in\Matrix{n}$ -- произвольная квадратная матрица и $\lambda \in\mathbb R$.
Рассмотрим функцию $\chi_A(\lambda) = \det(\lambda E - A)$.

\begin{claim}
\label{claim::CharSpec}
Пусть $A\in\Matrix{n}$.
Тогда верно
\begin{enumerate}
\item Функция $\chi_A(\lambda)$ является многочленом степени $n$ со старшим коэффициентом $1$.

\item Для произвольного числа $\lambda$ верно, что $\lambda \in\spec_\mathbb R A$ тогда и только тогда, когда $\chi_A(\lambda) = 0$.%
\footnote{Аналогичное утверждение верно и для $\spec_\mathbb C A$.}
\end{enumerate}
\end{claim}
\begin{proof}
(1) Давайте посмотрим на явную формулу определителя
\[
\det B = \sum_{\sigma\in\Sym{n}}\sgn(\sigma) b_{1\sigma(1)}\ldots b_{n\sigma(n)}
\]
Заметим, что данное выражение является многочленом от коэффициентов матрицы $A$, причем все его слагаемые имеют степень $n$.
Теперь, когда мы считаем характеристический многочлен, мы находим $\det(\lambda E - A)$.
То есть вместо $b_{ii}$ мы должны подставить  $\lambda - a_{ii}$, а вместо $b_{ij}$ взять $-a_{ij}$ (при $i\neq j$).
То есть мы в многочлен от многих переменных подставляем либо числа, либо линейный многочлен от $\lambda$.
Понятно, что результат будет многочлен от $\lambda$ причем степени уж точно не больше $n$.
Теперь давайте поймем какая будет у него степень и старший коэффициент.
\[
\lambda E - A = 
\begin{pmatrix}
{\lambda - a_{11}}&{\ldots}&{-a_{1n}}\\
{\vdots}&{\ddots}&{\vdots}\\
{-a_{n1}}&{\ldots}&{\lambda - a_{nn}}\\
\end{pmatrix}
\]
Ясно, что максимальная степень по $\lambda$ может вылезти только из слагаемого являющегося произведением диагональных элементов -- $(\lambda - a_{11}) \ldots (\lambda - a_{nn})$.
А его старший член $\lambda^n$.
Вот и все.

(2) Вспомним, что $\lambda\in\spec_\mathbb R A$ тогда и только тогда, когда $A - \lambda E$ -- необратимая матрица или что то же самое,  $\lambda E - A$ -- необратимая матрица.
Матрица необратима тогда и только тогда, когда ее определитель ноль.
Потому $\lambda \in\spec_\mathbb R A$ тогда и только тогда, когда $\det (\lambda E - A) = 0$, то есть $\chi_A(\lambda) = 0$.
Что и требовалось.
\end{proof}

Для произвольной матрицы $A\in\Matrix{n}$ многочлен $\chi_A(\lambda)$ называется {\it характеристическим многочленом} матрицы $A$.

\paragraph{Явные формулы для коэффициентов характеристического многочлена}

Вначале давайте введем некоторые обозначения. Пусть $A\in\Matrix{n}$ -- некоторая матрица.
Рассмотрим произвольное $k$ элементное подмножество в множестве чисел от $1$ до $n$ заданное в виде $i_1,\ldots, i_k$%
\footnote{Здесь предполагается, что $i_1 < \ldots < i_k$.}
Вычеркнем из матрицы $A$ столбцы и строки с этими номерами и обозначим полученную матрицу через $R_{i_1,\ldots,i_k}$.
Графически эта процедура выглядит так:
\[
\begin{array}{cc}
{}&{
\begin{array}{ccccc}
{}&{i_1\phantom{aa}}&{\ldots}&{\phantom{aa}i_k}&{}\\
\end{array}
}\\
{
\begin{array}{c}
{}\\{i_1}\\{\vdots}\\{i_k}\\{}
\end{array}
}&{
\left(
\begin{array}{c|c|c|c|c}
\cline{2-2}
\cline{4-4}
{R_{1\,1}}&{
\begin{array}{c}
{a_{1i_1}}\\{\vdots}
\end{array}
}&{\ldots}&{
\begin{array}{c}
{a_{1i_k}}\\{\vdots}
\end{array}
}&{R_{1\,k+1}}\\
\hline
\multicolumn{1}{|c|}{
\begin{array}{cc}
{a_{i_11}}&{\ldots}\\
\end{array}
}&{}&{}&{}&\multicolumn{1}{c|}{
\begin{array}{cc}
{\ldots}&{a_{i_1n}}\\
\end{array}
}\\
\hline
{\vdots}&{}&{\ddots}&{}&{\vdots}\\
\hline
\multicolumn{1}{|c|}{
\begin{array}{cc}
{a_{i_k1}}&{\ldots}\\
\end{array}
}&{}&{}&{}&\multicolumn{1}{c|}{
\begin{array}{cc}
{\ldots}&{a_{i_kn}}\\
\end{array}
}\\
\hline
{R_{k+1\,1}}&{
\begin{array}{c}
{\vdots}\\{a_{ni_1}}
\end{array}
}&{\ldots}&{
\begin{array}{c}
{\vdots}\\{a_{ni_k}}
\end{array}
}&{R_{k+1\,k+1}}\\
\cline{2-2}
\cline{4-4}
\end{array}
\right)
}\\
\end{array}
\mapsto
R_{i_1,\ldots,i_k} = 
\begin{pmatrix}
{R_{1\,1}}&{\ldots}&{R_{1\,k+1}}\\
{\vdots}&{\ddots}&{\vdots}\\
{R_{k+1\,1}}&{\ldots}&{R_{k+1\,k+1}}\\
\end{pmatrix}
\in\Matrix{n-k}
\]
Пользуясь этими обозначениями покажем следующее.

\begin{claim}
Пусть $A\in\Matrix{n}$ и его характеристический многочлен имеет вид 
\[
\chi_A(\lambda) = \lambda^n + a_{n-1}\lambda^{n-1} + \ldots + a_1 \lambda + a_0
\]
Тогда
\begin{enumerate}
\item В обозначениях выше, для коэффициентов $a_k$ верна следующая формула%
\footnote{Заметим, что эта формула также имеет смысл при $k=0$ и при $k = n$.
Если $k = 0$, то множество индексов пусто $\varnothing$ и $R_\varnothing = A$, потому формула превращается в равенство $a_0 = (-1)^n\det A$.
При условии $ k = n$, мы вычеркиваем все строки из матрицы и в этом случае $R_{1,\ldots,n}\in\Matrix{0}$.
Такого объекта не существует, но мы можем для удобства считать, что в этом случае формула означает $\det R_{1,\ldots,n} = 1$.}
\[
a_{k} = (-1)^{n-k}\left(\sum_{i_1<\ldots<i_k}\det R_{i_1,\ldots,i_k}\right)
\]

\item $a_0 = (-1)^n\det A$.

\item $a_{n-1} =  - \tr A$.
\end{enumerate}
\end{claim}
\begin{proof}
(1) Введем обозначения для столбцов матрицы $A = (A_1|\ldots|A_n)$ и пусть $e_i\in\mathbb R^n$ -- столбец, у которого $i$-я координата равна $1$, а все остальные $0$.
Нам надо посчитать $\det(\lambda E - A) = (-1)^n \det(A-\lambda E)$.
Тогда, 
\[
\det(A-\lambda E) = \det(A_1 - \lambda e_1|\ldots|A_n - \lambda e_n)
\]
Теперь надо раскрыть последний определитель по полилинейности.%
\footnote{Думать про это выражение надо так: надо мысленно заменить вертикальные черточки умножением и считать, что мы раскрываем скобки в произведении.}
Всего у нас будет $2^n$ слагаемых, каждое из которых -- это определитель матрицы состоящей из столбцов $A_i$ или $-\lambda e_j$, стоящих вперемешку.

Давайте для определенности считать, что у нас $n=5$, тогда мы считаем
\[
\det(A_1-\lambda e_1|A_2-\lambda e_2|A_3 - \lambda e_3|A_4 -\lambda e_4|A_5 - \lambda A_5)
\]
Среди слагаемых давайте посмотрим на слагаемое, содержащее $2$ столбца матрицы $A$ и $3$ столбца вида $-\lambda e_i$, например, такое
\[
\det(A_1|-\lambda e_2|A_3|-\lambda e_4|-\lambda e_5) = 
\det
\begin{pmatrix}
{a_{11}}&{0}&{a_{13}}&{0}&{0}\\
{a_{21}}&{-\lambda}&{a_{23}}&{0}&{0}\\
{a_{31}}&{0}&{a_{33}}&{0}&{0}\\
{a_{41}}&{0}&{a_{43}}&{-\lambda}&{0}\\
{a_{51}}&{0}&{a_{53}}&{0}&{-\lambda}\\
\end{pmatrix}
\]
Давайте последовательно разлагать этот определитель по $2$-ому, $4$-ому и $5$-ому столбцам.
Обратим внимание, что $-\lambda$ всегда будут стоять на диагонали, потому знаки всех алгебраических дополнений будут положительными:
\[
\det
\begin{pmatrix}
{a_{11}}&{0}&{a_{13}}&{0}&{0}\\
{a_{21}}&{-\lambda}&{a_{23}}&{0}&{0}\\
{a_{31}}&{0}&{a_{33}}&{0}&{0}\\
{a_{41}}&{0}&{a_{43}}&{-\lambda}&{0}\\
{a_{51}}&{0}&{a_{53}}&{0}&{-\lambda}\\
\end{pmatrix}
=
(-\lambda)
\det
\begin{pmatrix}
{a_{11}}&{a_{13}}&{0}&{0}\\
{a_{31}}&{a_{33}}&{0}&{0}\\
{a_{41}}&{a_{43}}&{-\lambda}&{0}\\
{a_{51}}&{a_{53}}&{0}&{-\lambda}\\
\end{pmatrix}
=
(-\lambda)^2
\det
\begin{pmatrix}
{a_{11}}&{a_{13}}&{0}\\
{a_{31}}&{a_{33}}&{0}\\
{a_{51}}&{a_{53}}&{-\lambda}\\
\end{pmatrix}
=
(-\lambda)^3
\det
\begin{pmatrix}
{a_{11}}&{a_{13}}\\
{a_{31}}&{a_{33}}\\
\end{pmatrix}
\]
В общем случае слагаемое с $k$ столбцами вида $-\lambda e_i$ является определителем матрицы вида
\[
\begin{array}{cc}
{}&{
\begin{array}{ccccc}
{}&{i_1\phantom{aa}}&{\ldots}&{\phantom{aa}i_k}&{}\\
\end{array}
}\\
{
\begin{array}{c}
{}\\{i_1}\\{\vdots}\\{i_k}\\{}
\end{array}
}&{
\left(
\begin{array}{c|c|c|c|c}
\cline{2-2}
\cline{4-4}
{R_{1\,1}}&{
\begin{array}{c}
{0}\\{\vdots}
\end{array}
}&{\ldots}&{
\begin{array}{c}
{0}\\{\vdots}
\end{array}
}&{R_{1\,k+1}}\\
\hline
\multicolumn{1}{|c|}{
\begin{array}{cc}
{a_{i_11}}&{\ldots}\\
\end{array}
}&{-\lambda}&{}&{}&\multicolumn{1}{c|}{
\begin{array}{cc}
{\ldots}&{a_{i_1n}}\\
\end{array}
}\\
\hline
{\vdots}&{}&{\ddots}&{}&{\vdots}\\
\hline
\multicolumn{1}{|c|}{
\begin{array}{cc}
{a_{i_k1}}&{\ldots}\\
\end{array}
}&{}&{}&{-\lambda}&\multicolumn{1}{c|}{
\begin{array}{cc}
{\ldots}&{a_{i_kn}}\\
\end{array}
}\\
\hline
{R_{k+1\,1}}&{
\begin{array}{c}
{\vdots}\\{0}
\end{array}
}&{\ldots}&{
\begin{array}{c}
{\vdots}\\{0}
\end{array}
}&{R_{k+1\,k+1}}\\
\cline{2-2}
\cline{4-4}
\end{array}
\right)
}\\
\end{array}
=I_{i_1,\ldots,i_k}
\]
Раскладывая этот определитель по столбцам $i_1, \ldots, i_k$ мы получаем
\[
\det I_{i_1,\ldots,i_k} = (-\lambda)^k
\det R_{i_1,\ldots,i_k}
\]
Слагаемые при $\lambda^k$ вылезут, когда ровно $k$ столбцов имеют вид $-\lambda e_i$.
Остается не забыть, что мы считали $(-1)^n\chi_A(\lambda)$.

(2) Свободный член многочлена $\chi_A(\lambda)$ всегда равен $\chi_A(0) = \det(0 E - A) = \det(-A) = (-1)^n \det(A)$, что и требовалось.

(3) Для подсчета $a_{n-1}$ воспользуемся формулой, получим%
\footnote{Здесь $\hat i $ означает, что индекс $i$ пропущен.}
\[
a_{n-1} = (-1)^{n - (n-1)}\sum_{i=1}^n\det R_{1,\ldots,\hat i,\ldots,n}
\]
Но заметим, что $R_{1,\ldots, \hat i,\ldots, n} = a_{ii}$, а значит предыдущее равенство превращается в
\[
a_{n-1} = (-1)^{n - (n-1)}\sum_{i=1}^n a_{ii} = - \tr A
\]
\end{proof}

\paragraph{Примеры}

\begin{enumerate}
\item Если $A\in\Matrix{1}$, то есть $A = a\in\mathbb R$ -- число, то $\chi_A(\lambda) = \lambda - a$.

\item Если $A\in\Matrix{2}$, то $\chi_A(\lambda) = \lambda^2 - \tr A \lambda + \det A$.

\item Если $A\in\Matrix{3}$, то $\chi_A(\lambda) = \lambda^3 - \tr A \lambda^2 + a_1 \lambda - \det A$, где
\[
a_1 = 
\det
\begin{pmatrix}
{a_{22}}&{a_{23}}\\
{a_{32}}&{a_{33}}\\
\end{pmatrix}
+
\det
\begin{pmatrix}
{a_{11}}&{a_{13}}\\
{a_{31}}&{a_{33}}\\
\end{pmatrix}
+
\det
\begin{pmatrix}
{a_{11}}&{a_{12}}\\
{a_{21}}&{a_{22}}\\
\end{pmatrix}
\]
\end{enumerate}

Стоит отметить, что считать характеристические многочлены от матриц большего размера через эти формулы практически не целесообразно.
Максимальный разумный размер -- матрица $4$ на $4$.
Самый быстрый способ остается алгоритм Гаусса для подсчета определителя $\det(\lambda E - A)$ с символьными коэффициентами.



